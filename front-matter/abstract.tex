\begin{center}
	{\large \textbf{Abstract}}\\
	
	{\large \textbf{The EMC Effect in A=3 Nuclei}}\\
\end{center}

The European Muon Collaboration(EMC) discovered an unexpected and puzzling result in 1983 when comparing the deep inelastic scattering(DIS) nuclear structure functions of Deuterium and Iron. The larger than expected structure function for Deuterium compared to the large nuclei of  Iron was coined the EMC effect and has been studied for nearly four decades. Experiments at CERN, Stanford (SLAC), Thomas Jefferson National Accelerator Facility(JLab), and other labs have studied the EMC effect for a range of different nuclei, attempting to complete a part of the EMC puzzle. I will discuss recent results from JLab exploring the EMC effect by using an electron beam to probe two mirror nuclei, Helium-3 and Tritium. 
