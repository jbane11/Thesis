
\chapter{Conclusion}
\paragraph{}The MARATHON experiment took inclusive DIS data on the special designed sealed tritium cells, filled with tritium, helium, and deuterium at Thomas Jefferson National Accelerator Facility. The unique opportunity provided by access to tritium allowed the MARATHON experiment to be the first to use DIS to study the internal structurer of this radioactive A=3 system. Alongside tritium, DIS data was taken on both helium-3 and deuterium for ratio comparison. The experimental kinematics produces DIS events for $x$ from 0.18 to 0.8, with a Q$^2$ range of 3 to 12 (GeV$^2$) by rotating the electron spectrometer from 17.5 to 36 degrees and using the Continuous Electron Beam Accelerator Facility's 10.6 GeV beam. 
\paragraph{}The MARATHON data was used to produce the inclusive DIS cross section for these three gas targets. This measurement has provided the first extraction of the DIS cross section for tritium. Using the cross section measurements of the two A=3 mirror nuclei and deuterium, the MARATHON collaboration has calculated the A=3 EMC effect for both helium-3 and tritium. This experiment will provide the first ever results on the EMC effect of tritium and the first on the comparison of the EMC effects of the two A=3 mirror nuclei.
\paragraph{}