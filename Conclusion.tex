
\chapter{Conclusion}
\paragraph{}The MARATHON experiment took inclusive DIS data on the specially designed sealed $^3$H cells, filled with $^3$H, $^3$He, and $^2$D at the Thomas Jefferson National Accelerator Facility. The unique opportunity provided by access to $^3$H allowed the MARATHON experiment to be the first to use DIS to study the internal structure of this radioactive A=3 system. Alongside $^3$H, DIS data were taken on both $^3$He and $^2$D for ratio comparison. The experimental kinematics produces DIS events for $x$ from 0.18 to 0.8, with a Q$^2$ range of 3 to 12 (GeV$^2$) by rotating the electron spectrometer from 17.5 to 36 degrees and using the Continuous Electron Beam Accelerator Facility's 10.6 GeV beam. 
\paragraph{}The MARATHON data was used to produce results for the inclusive DIS cross section for these three gas targets. This measurement has provided the first extraction of the DIS cross section for $^3$H. Using the cross section measurements of the two A=3 mirror nuclei and $^2$D, the MARATHON collaboration has calculated the A=3 EMC effect for both $^3$He and $^3$H. This experiment will provide the first ever results on the EMC effect of $^3$H and the first on the comparison of the EMC effects of the two A=3 mirror nuclei. The MARATHON EMC effect for $^3$He agrees well with the previous JLab EMC measurement from experiment E03103 J. Seely and A. Daniels \cite{seeley}. Due to this agreement, my analysis methods for the extraction of the EMC effect for $^3$He are validated, and therefore are valid for the extraction the EMC effect for $^3$H. The measurement of both the $^3$He and $^3$H EMC effects are important because comparison of the EMC measurements between these two A=3 nuclei can help evaluate isospin effects, and help remove model dependence of nuclear effects in the extraction of the $F^n_2/F^p_n$ structure function ratio. 

\paragraph{}The main goal of my analysis was to study the comparison of the EMC effect for the two A=3 mirror nuclei. I show the comparison of the EMC effect for $^3$H to the EMC effect for $^3$He in figure \ref{ISORatio}. This comparison shows no difference from unity within the precision of the current status of the analysis. 

\paragraph{}The biggest sources of error for the EMC effect are the isoscalar correction and cross section model dependence. The MARATHON's measurement of the $F^n_2/F^p_n$ ratio will help improve the errors associated with the isoscalar correction once the MARATHON collaboration finishes the analysis. The $F^n_2/F^p_n$ produced by a ratio of $^3$H and $^3$He will greatly reduce the error of the structure function ratio at high values of $x$ due to the small differences in nuclear effects in comparing the two A=3 mirror systems. A better understanding of the nucleon structure functions and the EMC effect will reduce the errors associated with the cross section extraction using a cross section model. Using an iterative procedure would help reduce the magnitude of the errors caused by the model dependence. A continuing goal of my analysis is to introduce an iterative procedure to correct the cross section model with data extracted cross sections. 

\paragraph{}The aim of this analysis is not to solve the EMC puzzle but to help find a part of the solution. The measurement of the EMC effect of $^3$H will add to the pool of previously measured nuclei that will continue to grow. A proposed experiment at Jefferson Lab plans to expand the database of known EMC effects by measuring the EMC effect on a large range of light and heavy nuclei \cite{pro_gaskell}. Another planned experiment at Jefferson Lab was proposed to provide new constraints on the EMC effect models by measuring the spin-dependent EMC effect on polarized $^7$Li \cite{pro_brooks}. Also, the flavor dependence of the EMC effect could be studied via pion induced Drell-Yan scattering\cite{Dutta} or tagged deep inelastic scattering measurements using the ALERT (Low Energy Recoil Tracker) detector in Hall B\cite{Armstrong}.


