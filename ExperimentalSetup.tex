


\chapter{ Experimental Setup}

\section{Thomas Jefferson Lab}
\paragraph{}Thomas Jefferson Lab (Jlab) in Newport News, Virgina hosted the MARATHON experiment in the Fall of 2017 and Spring of 2018. Jlab uses support from the U.S. Department of Energy(DOE) and the state of Virgina to complete the lab's mission of delivering productive research by exploring the atomic nucleus and its fundamental constituents, including precise tests of their interactions. Along with applying an advanced particle accelerator, particle detectors and other technologies to develop new basic research capabilities and to address the challenges of a modern society.
	\subsection{CEBAF}
	\paragraph{}The Continuous Electron Beam Accelerator Facility (CEBAF) was recently upgraded to a 12 GeV accelerator, upgrading it to be able to supply a 11 GeV beam of continuous electrons of up to 200 $\mu$A of current to three experimental halls (A,B,C) and 12 GeV to the recently constructed hall D. After being accelerated to 45 MeV by a polarized electron gun or a thermionic injector, the electrons are injected into the North linear accelerator (LINAC), shown in figure \ref{CEBAF}. The polarized gun can supply electrons with up to 80$\%$ polarization and the polarization direction can be controlled by a wien filter. To ensure the level of polarization, a 5 MeV Mott polarimeter may be used to measure the level of polarization\cite{HallA}.
	\paragraph{} The electrons are conveyed through two LINACs and two bending arcs per complete pass of the accelerator. Electrons traveling to Halls A, B, and C complete a maximum of four and a half revolutions around the accelerator. Electrons going to all D travel through the north LINAC for an extra boost. These particles receive approximately 2.2 GeV in energy for each cycle through the accelerator. The radio frequency (RF) cavities in each LINAC use an oscillating electromagnetic field to supply a force to accelerate the passing electrons. These Niobium RF cavities are cooled to 2 K in order to create conditions that allow the cavities to be superconducting \cite{HallA}.    
	
	\begin{figure}[h]
	\centering
	 \caption{Schematic Layout of CEBAF. }
	 \label{CEBAF}
	 \includegraphics[width=10cm]{CEBAF.png} 
	 \end{figure} 
	 
	 \subsection{Hall A}
	 
	\begin{figure}[H]
		\centering
		\caption{A 3D drawing of Hall A. }
		\label{HallA}
		\includegraphics[width=14cm]{HallA_2.png} 
	\end{figure} 	 
	 
	 \paragraph{}The experimental Hall A and the scientific equipment used were designed for detailed investigations of the internal structure of nuclei. Two high resolution spectrometers in Hall A use the inclusive (e,e$\prime$) and exclusive (e,e$\prime$ p) reactions to gain a greater understanding of the structure of the nucleus. Completing detailed studies with high resolution and extreme accuracy requires knowing the beam position, size, energy, current, direction, and polarization when the beam strikes the target. The instrumentation used in the precise measurement of these quantities in Hall A  are shown in figure \ref{BeamLine} \cite{HallA}.

	 \paragraph{} A pair of Beam Position Monitors(BPM)s are used to measure the relative beam position without affecting the beam. The two Hall A BPMs are located at 7.524 m and 1.286 m away from the target. Using the standard difference-over-sum technique, the relative beam position is determined with an accuracy of 100 $\mu$m with a beam current of at least 1 $\mu$A \cite{HallA}. The BPMs' positional data is recorded in two ways. Every second of beam time, the beam position average over 0.3 seconds is logged into the Experimental Physics and Industrial Control System (EPICS) database. The BPMs also transmit data event-by-event to the CEBAF online Data Acquisition system(CODA).
	 	 	 
 	 	\begin{figure}[H]
 	 		\centering
 	 		\caption{A schematic layout of the beam line in Hall. \cite{HallA} }
	 	 	\label{BeamLine}
	 	 	\includegraphics[width=14cm]{BeamLine.png} 
	 	 \end{figure} 	
	 	 	
	 \paragraph{} The main beam line components of the BPMs consist of four open-ended antennas. Figure \ref{BPMimg} shows a BPM chamber and figure \ref{BPM_4} shows the layout of the four antennas as you look down the beam line. In this chamber, the design of three of the four antennas can be seen. The antennas are titled $u_+$, $u_-$ and $v_+$, $v_-$. The antennas receive an induced signal as electrons pass to determine the beam position in the u and v directions. The direction of the beam is determined by using the two BPMs in conjunction with timing information provided. The accuracy of the BPMs requires an absolute measurement of the electron beam's position to calibrate the BPMs and a internal input oscillation measurement names twiddle to supply BPM signal coefficients.  \cite{BPM,BPM2}.
	 	 	\begin{figure}[H]
	 	 		\centering
	 	 		\caption{BPM design diagram, from JLab instrumentation	group. Beam direction is from left to right \cite{BPM2}. }
	 	 		\label{BPMimg}
	 	 		\includegraphics[width=10cm]{BPM.png} 
	 	 	\end{figure} 	
	 
	 		\begin{figure}[H]
	 			\centering
	 			\caption{BPM design diagram, looking down the beam line\cite{BPM2}. }
	 			\label{BPM_4}
	 			\includegraphics[width=10cm]{BPM_4.png} 
	 		\end{figure} 

	
	 \paragraph{} Damage to a target system from intense beam can cause extreme fluctuations in the target's temperature and density. A raster was used to counteract the damage caused by a focused beam. The raster used two magnetic fields produced by two dipoles to spread the electron beam out. This produces a large rectangle interaction area on the front face of the target container. A triangle wave of 25 kHz was used to control the coils of the dipole magnets. The raster systems are located $\approx$17 meters before the target chamber (upstream of the target\cite{BPM2}). The rasters position can be seen in figure \ref{HallA}. Safety constraints administrated by the target group at JLAB limited the minimum size of the raster spot for the MARATHON experiment to two millimeters by two millimeters. This limit was installed has a safety concern for the tritium target. 
	 \paragraph{} The Hall A raster system consists of four dipoles. Two dipoles produce magnetic fields in the horizontal direction of the lab frame and two in the vertical. The upstream raster and downstream rasters include one vertical and one horizontal dipole. The relative change in position of the incoming electrons are controlled by the current supplied to the dipoles. In order to obtain the change in beam position due to the rasters, a calibration between the raster current and measured beam position were obtained.  
	 
	 \paragraph{}The electron beam energy is located in many of the equations used in an electron scattering experiment. This can cause a noticeable increase in systematic error if the beam energy measurement is not made precisely. At JLAB for the MARATHON experiment, the beam energy was measured in two ways. In Hall A, the beam energy was measured by using the (e,e$\prime$p) method. On the beam line, 17 meters upstream from the target an ep scattering chamber is located. The beam was directed into the target containing a rotating 10-30 $\mu$m thick tape of C$H_2$. The scattering angle of the electron and the recoil angle of the proton are used to determine the beam energy using equation \ref{EP}. Where $M_p$ is the mass of the proton and $\theta_p, \theta_e$ are the scattered angle of the proton, electron respectively. 
	\begin{equation}
	\label{EP}
	E = Mp \frac{cos\theta_e + \frac{sin\theta_e}{tan\theta_p}-1}{1 - cos\theta_e} 
	\end{equation}
	The beam energy was also measured using the ark measurement method \cite{Flay}. This method uses changes is beam position and precise measurements of the magnetic fields around the beam line to determine the energy of the electron beam. The angle at which the electrons are bent through is related to the momentum of the electrons,
	\begin{equation}
	\label{arc}
	p = k \frac{\int \vec{B} \cdot d\vec{l}}{\theta}.
	\end{equation}	
	In equation \ref{arc}, p is the momentum of the electrons, $\theta$ is the bend angle, and $\vec{B}$ is the magnetic field the electron experiences. Then using the momentum of the electron, the energy of the beam can be extracted. The error on the beam energy measurement is $\delta$ E/E $\approx$ 2 $* 10^{-4} $ \cite{EPMet, Flay}.  The MARATHON experiment used both methods to accurately determine the electron beam energy.
	
		  	\begin{figure}[H]
		  	 	 		\centering
		  	 	 		\caption{Hall A Current Monitor components \cite{BCM1}. }
		  	 	 		\label{BCMpng}
		  	 	 		\includegraphics[width=10cm]{BCM1.png} 
		  	\end{figure}
	
	\paragraph{} The main process of measuring the scattering yield for a calculation of a cross section looks at finding the ratio of the number of electrons scattered to the number of electrons sent. In order to accurately determine the number of electrons sent to scatter with our target system, Hall A use a highly accurate and non-invasive beam current monitor(BCM). The Hall A BCM has an a4bsolute accuracy of 0.2 percent as long as the current is between 1 and 180 $\mu$A. The BCM used in Hall A consists of three main components: a Parametric Current Transformer (PCT) and two pill box cavities. Figure \ref{BCMpng} shows the components in the Hall A BCM.  The BCM produces an RF signal that is proportional to the beam current. An 10 kHz down converter, RMS-to-DC converter, voltage-to-Frequency converter, and a scaler are used to inject the current signal into the Hall A DAQ. Proportionality constants are determined in the calibration process to correctly integrate the charge for a given amount of beam current\cite{BCM1}. Continue after the initial beam line components, an electron will enter into the target chamber, housing the target system.
	  
\section{Target}
\paragraph{} The Hall A Tritium Target(HATT) system was used for the Tritium run group of experiments. The HATT target chamber was repurposed from a previously used cryo-target chamber in order to reduce the financial cost of designing a new target chamber. The refurbishing of the cryo-target chamber consisted of adding in new safety features to prevent and mitigate a tritium leak.  A 4 inch long collimator with an inner diameter of 0.4 inch was added inside of the target chamber but upstream of the target ladder to prevent the beam from striking the thin side wall of the aluminum cell. In case of a tritium leak in the target chamber, an exhaust system was installed to control the amount of tritium exposed to the Hall.\cite{HATT_eng}  Figure \ref{HATT} shows the HATT system with the target ladder in the home position and the scattering windows removed. 
\begin{figure}
	\caption{Target Images}
	\subfloat[A image of the HATT. \cite{DHimages}]{{\includegraphics[width=6cm]{HATT.jpg} }}
	\quad
	\subfloat[Image of the Hall A Tritium Target Ladder. \cite{DHimages}]{{\includegraphics[width=12cm]{HATT_Ladder.JPG} }}
	\label{HATT}


\end{figure}
A picture of the HATT ladder installed in the HATT system is shown if figure \ref{HATT}. The ladder contains both gaseous cells and solid targets. The MARATHON experiment had five gas cells. The top four of the gas cells were filled with tritium, deuterium, hydrogen, and $^3$helium, from top to bottom respectively. Due to safety restricts the tritium cell was not installed until the HATT system could be closed. The bottom most cell was left empty, to complete end cap subtractions. The lower half of the target ladder contains the solid targets used during the MARATHON experiment. Listed from top to bottom, the solid targets used were a pair of thick aluminum foils, carbon multifoil, single carbon foil, and a carbon foil with a 2mm diameter hole. The thick Al foils were used to aid the target window background subtraction. The multifoil target also know has the optics target was used to calibrate the z-axis  reconstruction of the optics matrix. The single carbon foil and carbon hole were used to calibrate the BPMs and rasters and to determine the off set of the central line of the detector. 



\section{High Resolution Spectrometers}
Electrons that successfully scatter from the target may end up in either of the two HRSs(High Resolution Spectrometers). The HRSs were designed to detect charged particles with a high degree of precision. 
In order to achieve a high level of resolution in momentum and angle, the HRSs were designed with a magnet configuration of QQ$D_n$Q (quadrupole, quadrupole, dipole, and quadrupole). The vertical bending dipole provides the field required to transport the scattered particles through the 45$^\circ$ bending angle to the detector hut. A drawing of an HRS can be seen in figure \ref(hrsfull). The first quadrupole(Q1) focuses the incoming electrons in the vertical plane. The following two quadrupoles (Q2 and Q3 provide transverse focusing. This optical design allows the use of extended gas targets with no substantial loss in solid angle\cite{HallA}.  The spectrometers were designed to perform various functions which include: triggering the data acquisition system (DAQ) when certain requirements are met, gathering the position and direction of individual particles to reconstruct a track, provide precise timing information, and identify many different particle types that pass through the detector system. Both the Left and Right HRSs contain two planes of Scintillators to function has the main trigger for the detector package. The vertical drift chambers (VDC) that lay at the front of the detector in conjunction with the Shower that lies in the back of the detector provide information for reconstructing the particle tracks and precise timing. Particles are identified by the Cherenkovs, shower calorimeters, and Pion Rejectors that are contained in the left or right HRS. The layout of the individual detectors that make up the left and right detector package are shown in figure \ref{hrsss}  \cite{HallA}.

\begin{figure}
	\centering
	\includegraphics[width=10cm]{HRS_full.png}
	\caption{A side view of a HRS \cite{HallA}.
	\label{hrsfull}}
\end{figure}

\begin{figure}
	\centering
	\includegraphics[width=10cm]{HRSs.png}
	\caption{A view of both the left(top) and right(bottom) detector stacks inside the left and right HRS \cite{HallA}.
	\label{hrsss}}
\end{figure}





	\subsection{Vertical Drift Chambers}
	Each of the spectrometers housed in Hall A contains a vertical drift chamber(VDC). Each VDC contains two planes of crossing sense wires. Shown in figure \ref{VDC_profile}, the two planes of the VDC lie a distance of 0.335m apart \cite{drift}. The lower plane of the VDC is positioned at the approximate focal plane of the HRS and lies in the horizontal plane of the Hall A coordinate system. The sense wires located in the VDCs cross orthogonally. They are offset by $45^\circ$ in respect to the dispersive and non-dispersive directions. 
	
	\cite{drift}
	
	
	\begin{figure}
	\centering
	
	\includegraphics[width=10cm]{VDC_profile_view.png}
	
	\caption{A sketch of the two VDC planes in the HRSs with a particle traveling through the detector at 45$^\circ$.\cite{drift}.
	\label{VDC_profile}}
	\end{figure}
	
	
	
	
	
	\subsection{Scintillators}	\subsection{Cherenkov}
	\subsection{Shower Calorimeter}
	\subsection{Pion Rejector}
	\subsection{FPP Chambers}

\section{Trigger Setup}
\section{DAQ - Data Acquisition System}

\section{Kinematic Settings}









